\documentclass[11pt]{lettre}
 
\usepackage[utf8]{inputenc}
\usepackage[T1]{fontenc}
\usepackage{lmodern}
\usepackage{eurosym}
\usepackage[frenchb]{babel}
\usepackage{numprint}
% Suppression du trait de pliage
\makeatletter
\newcommand*{\NoRule}{\renewcommand*{\rule@length}{0}}
\makeatother 

\usepackage[dvipdfm]{hyperref}
\hypersetup{
    pdftitle={Titre},    % title
    pdfauthor={Sovannara HAK},     % author
    pdfsubject={Sujet},   % subject of the document
    pdfcreator={LaTeX},   % creator of the document
    pdftoolbar=false,        % show Acrobat’s toolbar?
    pdfmenubar=false,        % show Acrobat’s menu?
}
\begin{document}
 
\begin{letter}{Destinataire \\ 
        adresse 1 \\ 
		adresse 2\\ 
        adresse 3}
\NoRule    
\name{Sovannara Hak}
\address{Sovannara Hak\\76 Boulevard Ornano\\
        75018 Paris\\
        06 81 13 47 04\\
        hak.sovannara@gmail.com}
\lieu{Paris}
\nofax
\notelephone
 
\vspace{-4cm}
\def\concname{Objet :~} % On définit ici la commande 'objet'
\conc{L'objet}
\opening{Madame, Monsieur}
 
J'aime écrire les lettres en \LaTeX.
 
\closing{Cordialement,}
 
\encl{Document 1}
\end{letter}
 
\end{document}
